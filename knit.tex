% Options for packages loaded elsewhere
\PassOptionsToPackage{unicode}{hyperref}
\PassOptionsToPackage{hyphens}{url}
%
\documentclass[
]{article}
\usepackage{amsmath,amssymb}
\usepackage{iftex}
\ifPDFTeX
  \usepackage[T1]{fontenc}
  \usepackage[utf8]{inputenc}
  \usepackage{textcomp} % provide euro and other symbols
\else % if luatex or xetex
  \usepackage{unicode-math} % this also loads fontspec
  \defaultfontfeatures{Scale=MatchLowercase}
  \defaultfontfeatures[\rmfamily]{Ligatures=TeX,Scale=1}
\fi
\usepackage{lmodern}
\ifPDFTeX\else
  % xetex/luatex font selection
\fi
% Use upquote if available, for straight quotes in verbatim environments
\IfFileExists{upquote.sty}{\usepackage{upquote}}{}
\IfFileExists{microtype.sty}{% use microtype if available
  \usepackage[]{microtype}
  \UseMicrotypeSet[protrusion]{basicmath} % disable protrusion for tt fonts
}{}
\makeatletter
\@ifundefined{KOMAClassName}{% if non-KOMA class
  \IfFileExists{parskip.sty}{%
    \usepackage{parskip}
  }{% else
    \setlength{\parindent}{0pt}
    \setlength{\parskip}{6pt plus 2pt minus 1pt}}
}{% if KOMA class
  \KOMAoptions{parskip=half}}
\makeatother
\usepackage{xcolor}
\usepackage[margin=1in]{geometry}
\usepackage{color}
\usepackage{fancyvrb}
\newcommand{\VerbBar}{|}
\newcommand{\VERB}{\Verb[commandchars=\\\{\}]}
\DefineVerbatimEnvironment{Highlighting}{Verbatim}{commandchars=\\\{\}}
% Add ',fontsize=\small' for more characters per line
\usepackage{framed}
\definecolor{shadecolor}{RGB}{248,248,248}
\newenvironment{Shaded}{\begin{snugshade}}{\end{snugshade}}
\newcommand{\AlertTok}[1]{\textcolor[rgb]{0.94,0.16,0.16}{#1}}
\newcommand{\AnnotationTok}[1]{\textcolor[rgb]{0.56,0.35,0.01}{\textbf{\textit{#1}}}}
\newcommand{\AttributeTok}[1]{\textcolor[rgb]{0.13,0.29,0.53}{#1}}
\newcommand{\BaseNTok}[1]{\textcolor[rgb]{0.00,0.00,0.81}{#1}}
\newcommand{\BuiltInTok}[1]{#1}
\newcommand{\CharTok}[1]{\textcolor[rgb]{0.31,0.60,0.02}{#1}}
\newcommand{\CommentTok}[1]{\textcolor[rgb]{0.56,0.35,0.01}{\textit{#1}}}
\newcommand{\CommentVarTok}[1]{\textcolor[rgb]{0.56,0.35,0.01}{\textbf{\textit{#1}}}}
\newcommand{\ConstantTok}[1]{\textcolor[rgb]{0.56,0.35,0.01}{#1}}
\newcommand{\ControlFlowTok}[1]{\textcolor[rgb]{0.13,0.29,0.53}{\textbf{#1}}}
\newcommand{\DataTypeTok}[1]{\textcolor[rgb]{0.13,0.29,0.53}{#1}}
\newcommand{\DecValTok}[1]{\textcolor[rgb]{0.00,0.00,0.81}{#1}}
\newcommand{\DocumentationTok}[1]{\textcolor[rgb]{0.56,0.35,0.01}{\textbf{\textit{#1}}}}
\newcommand{\ErrorTok}[1]{\textcolor[rgb]{0.64,0.00,0.00}{\textbf{#1}}}
\newcommand{\ExtensionTok}[1]{#1}
\newcommand{\FloatTok}[1]{\textcolor[rgb]{0.00,0.00,0.81}{#1}}
\newcommand{\FunctionTok}[1]{\textcolor[rgb]{0.13,0.29,0.53}{\textbf{#1}}}
\newcommand{\ImportTok}[1]{#1}
\newcommand{\InformationTok}[1]{\textcolor[rgb]{0.56,0.35,0.01}{\textbf{\textit{#1}}}}
\newcommand{\KeywordTok}[1]{\textcolor[rgb]{0.13,0.29,0.53}{\textbf{#1}}}
\newcommand{\NormalTok}[1]{#1}
\newcommand{\OperatorTok}[1]{\textcolor[rgb]{0.81,0.36,0.00}{\textbf{#1}}}
\newcommand{\OtherTok}[1]{\textcolor[rgb]{0.56,0.35,0.01}{#1}}
\newcommand{\PreprocessorTok}[1]{\textcolor[rgb]{0.56,0.35,0.01}{\textit{#1}}}
\newcommand{\RegionMarkerTok}[1]{#1}
\newcommand{\SpecialCharTok}[1]{\textcolor[rgb]{0.81,0.36,0.00}{\textbf{#1}}}
\newcommand{\SpecialStringTok}[1]{\textcolor[rgb]{0.31,0.60,0.02}{#1}}
\newcommand{\StringTok}[1]{\textcolor[rgb]{0.31,0.60,0.02}{#1}}
\newcommand{\VariableTok}[1]{\textcolor[rgb]{0.00,0.00,0.00}{#1}}
\newcommand{\VerbatimStringTok}[1]{\textcolor[rgb]{0.31,0.60,0.02}{#1}}
\newcommand{\WarningTok}[1]{\textcolor[rgb]{0.56,0.35,0.01}{\textbf{\textit{#1}}}}
\usepackage{graphicx}
\makeatletter
\def\maxwidth{\ifdim\Gin@nat@width>\linewidth\linewidth\else\Gin@nat@width\fi}
\def\maxheight{\ifdim\Gin@nat@height>\textheight\textheight\else\Gin@nat@height\fi}
\makeatother
% Scale images if necessary, so that they will not overflow the page
% margins by default, and it is still possible to overwrite the defaults
% using explicit options in \includegraphics[width, height, ...]{}
\setkeys{Gin}{width=\maxwidth,height=\maxheight,keepaspectratio}
% Set default figure placement to htbp
\makeatletter
\def\fps@figure{htbp}
\makeatother
\setlength{\emergencystretch}{3em} % prevent overfull lines
\providecommand{\tightlist}{%
  \setlength{\itemsep}{0pt}\setlength{\parskip}{0pt}}
\setcounter{secnumdepth}{-\maxdimen} % remove section numbering
\ifLuaTeX
  \usepackage{selnolig}  % disable illegal ligatures
\fi
\usepackage{bookmark}
\IfFileExists{xurl.sty}{\usepackage{xurl}}{} % add URL line breaks if available
\urlstyle{same}
\hypersetup{
  pdftitle={home\_work},
  pdfauthor={Zarva\_I},
  hidelinks,
  pdfcreator={LaTeX via pandoc}}

\title{home\_work}
\usepackage{etoolbox}
\makeatletter
\providecommand{\subtitle}[1]{% add subtitle to \maketitle
  \apptocmd{\@title}{\par {\large #1 \par}}{}{}
}
\makeatother
\subtitle{DEAD\textbar LINE}
\author{Zarva\_I}
\date{2024-09-17}

\begin{document}
\maketitle

\#\#START

\subsection{R Markdown}\label{r-markdown}

This is an R Markdown document. Markdown is a simple formatting syntax
for authoring HTML, PDF, and MS Word documents. For more details on
using R Markdown see \url{http://rmarkdown.rstudio.com}.

When you click the \textbf{Knit} button a document will be generated
that includes both content as well as the output of any embedded R code
chunks within the document. You can embed an R code chunk like this:

\begin{Shaded}
\begin{Highlighting}[]
\FunctionTok{summary}\NormalTok{(cars)}
\end{Highlighting}
\end{Shaded}

\begin{verbatim}
##      speed           dist       
##  Min.   : 4.0   Min.   :  2.00  
##  1st Qu.:12.0   1st Qu.: 26.00  
##  Median :15.0   Median : 36.00  
##  Mean   :15.4   Mean   : 42.98  
##  3rd Qu.:19.0   3rd Qu.: 56.00  
##  Max.   :25.0   Max.   :120.00
\end{verbatim}

\subsection{Including Plots}\label{including-plots}

You can also embed plots, for example:

\includegraphics{knit_files/figure-latex/pressure-1.pdf}

Note that the \texttt{echo\ =\ FALSE} parameter was added to the code
chunk to prevent printing of the R code that generated the plot.

\#\#Domashka

citation(``ggplot2'') .libPaths() library()
source(``\url{https://bioconductor.org/biocLite.R}'')

\#\#\#Bioconductor

\begin{Shaded}
\begin{Highlighting}[]
\ControlFlowTok{if}\NormalTok{ (}\SpecialCharTok{!}\FunctionTok{require}\NormalTok{(}\StringTok{"BiocManager"}\NormalTok{, }\AttributeTok{quietly =} \ConstantTok{TRUE}\NormalTok{))  }\CommentTok{\#установил Bioconductor}
  \FunctionTok{install.packages}\NormalTok{(}\StringTok{"BiocManager"}\NormalTok{)}

\NormalTok{BiocManager}\SpecialCharTok{::}\FunctionTok{install}\NormalTok{(}\StringTok{"BiocVersion"}\NormalTok{)}
\end{Highlighting}
\end{Shaded}

\begin{verbatim}
## Bioconductor version 3.19 (BiocManager 1.30.23), R 4.4.1 (2024-06-14 ucrt)
\end{verbatim}

\begin{verbatim}
## Warning: package(s) not installed when version(s) same as or greater than current; use
##   `force = TRUE` to re-install: 'BiocVersion'
\end{verbatim}

\begin{verbatim}
## Installation paths not writeable, unable to update packages
##   path: C:/Program Files/R/R-4.4.1/library
##   packages:
##     boot, foreign, MASS, nlme, survival
\end{verbatim}

\begin{verbatim}
## Old packages: 'BiocManager', 'cpp11', 'curl', 'data.table', 'digest', 'ps',
##   'ragg', 'rmarkdown', 'tinytex', 'xfun'
\end{verbatim}

\#\#\#dplyr

\begin{Shaded}
\begin{Highlighting}[]
\CommentTok{\#mutate() — изменяет переменные, добавляет новые;}
\CommentTok{\#select() — выбирает переменные;}
\CommentTok{\#filter() — фильтрует объекты по условиям;}
\CommentTok{\#summarise() — вычисляет сводные статистики;}
\CommentTok{\#arrange() — сортирует по переменным;}
\CommentTok{\#group\_by() — группирует по значениям переменных;}
\CommentTok{\#*\_join() — группа глаголов для склеивания двух таблиц по ключу.}

\CommentTok{\#Пакет позволяет:}
\CommentTok{\#Поворачивать данные, то есть, преобразовывать их в длинный и широкий форматы: \#pivot\_longer(), pivot\_wider();}
\CommentTok{\#Разворачивать данные из вложенных списков в простые таблицы: unnest\_longer(), unnest\_wider();}
\CommentTok{\#Наоборот, делать из таблиц вложенные переменные: nest(), unnest();}
\CommentTok{\#Разделять и объединять столбцы по разделителю строк: separate(), unite();}
\CommentTok{\#Заполнять отсутствующие значения определёнными значениями или удалять их: complete(), drop\_na(), fill(), replace\_na().}
\end{Highlighting}
\end{Shaded}

\#\#\#readr, readxl, haven

\begin{Shaded}
\begin{Highlighting}[]
\CommentTok{\#Три основных пакета, которые помогают читать данные.}

\CommentTok{\#readr предназначен для чтения самых распространённых форматов данных: *.csv, *.txt, *.tsv;}
\CommentTok{\#redxl помогает читать файлы *.xlsx или, иными словами, всё, что пересылается в формате Excel;}
\CommentTok{\#haven читает форматы *.sas7bdat, *.sap, *.dta, *.sav, *.por, то есть данные из SAS.}
\end{Highlighting}
\end{Shaded}

\#\#\#purrr

\begin{Shaded}
\begin{Highlighting}[]
\CommentTok{\#Этот пакет может поначалу казаться сложным для понимания, однако на самом деле он просто расширяет функционал уже известного семейства функций *apply().}
\CommentTok{\#Функция map() и её расширения позволяют итерироваться по элементам векторов или списков самыми разными способами}
\end{Highlighting}
\end{Shaded}

\#\#\#tibble

\begin{Shaded}
\begin{Highlighting}[]
\CommentTok{\#В базовом R уже есть data.frame, однако tibble позволяет создавать гораздо более приятные для работы таблицы данных.}

\CommentTok{\#В отличие от data.frame он не приводит строки к факторам автоматически. Это нужно сделать самостоятельно, однако таким образом мы всегда в точности знаем, что находится в каждой переменной датафрейма;}
\CommentTok{\#Имена переменных остаются такими же, какими были. Например, в переменной с названием "variable name" пробел не будет заменён на точку, вместо этого имя будет окружено апострофами (обычно находятся на букве "ё"), что сохраняет ожидания от названий;}
\CommentTok{\#tibble оценивает переменные лениво. По существу, это значит, что при создании мы можем объявлять одну переменную на основе другой (но эта другая должна быть указана первой).}
\end{Highlighting}
\end{Shaded}

\#\#\#ggplot2 и ggpubr

\begin{Shaded}
\begin{Highlighting}[]
\CommentTok{\#Для визуализации данных существует сразу два пакета: базовый ggplot, содержащий в себе почти всю необходимую для создания графики функциональность.}
\CommentTok{\#Дополнительные интересные особенности вроде автоматического расчёта и добавления на график p{-}value, реализованы в пакете ggpubr.}
\end{Highlighting}
\end{Shaded}

\#\#\#flextable

\begin{Shaded}
\begin{Highlighting}[]
\CommentTok{\#Необходимо ещё и напечатать эти таблицы в виде, пригодном для восприятия коллегами без лишних проблем. Именно для этого нужен пакет flextable}
\end{Highlighting}
\end{Shaded}

\#\#\#stringr

\begin{Shaded}
\begin{Highlighting}[]
\CommentTok{\#Очень часто в данных есть строковые переменные, в которых записана некая текстовая информация, важная нам для исследования. Например, в одной ячейке записаны все возможные названия препарата, и стоит задача взять только тех участников исследования, у которых название препарата содержит подстроку "циклин".}
\end{Highlighting}
\end{Shaded}

\#\#\#lubridate

\begin{Shaded}
\begin{Highlighting}[]
\CommentTok{\#Помимо строковых данных часто встречаются и даты. С ними нельзя работать как с факторами, строками или, тем более, числами. Для этого существует специальная библиотека, которая может даже вычислить разницу между двумя датами с учётом високосных годов. }
\end{Highlighting}
\end{Shaded}

\#\#\#DescTools, psych

\begin{Shaded}
\begin{Highlighting}[]
\CommentTok{\#Существуют пакеты, в которых собраны статистические функции, позволяющие чуть ли не в одну строку подготовить базовый статистический отчёт.}
\CommentTok{\#В DescTools мы найдём полезные функции для статистических тестов и доверительных интервалов, а в psych функции для корреляционных матриц и расчёта сразу группы статистик.}
\end{Highlighting}
\end{Shaded}

\#\#\#read.*

\begin{Shaded}
\begin{Highlighting}[]
\CommentTok{\#Выше мы видим пример файла csv с разделителем{-}запятой. Обсудим конкретно, какими функциями читается каждый формат:}

\CommentTok{\#read.csv() читает csv с запятой в качестве разделителя;}
\CommentTok{\#read.csv2() читает csv с точкой с запятой в качестве разделителя (кстати, этот же формат отлично читает Excel, сразу разбивая его на столбцы);}
\CommentTok{\#read.tsv() читает csv со знаком табуляции в качестве разделителя (часто этот формат сохраняют в файле с расширением .txt).}
\CommentTok{\#Однако, в каждой функции при этом можно указать параметры sep, quote, dec, которые, соответственно, устанавливают знаки: разделителя, кавычек, десятичного разделителя. }
\end{Highlighting}
\end{Shaded}

\#\#\#write.*

\begin{Shaded}
\begin{Highlighting}[]
\CommentTok{\#write.csv(), write.csv2() печатают датафрейм в файлы csv с соответствующими разделителями, но их использование (особенно на Windows) может приводить к забавной вещи — поломке кодировки, когда символы в Excel не читаются человеком. Как решить эту проблему, мы узнаем в следующих шагах.}
\end{Highlighting}
\end{Shaded}

\begin{Shaded}
\begin{Highlighting}[]
\FunctionTok{read\_tsv}\NormalTok{(}\StringTok{"data/raw/data\_tsv.tsv"}\NormalTok{, }\AttributeTok{skip =} \DecValTok{0}\NormalTok{, }\AttributeTok{n\_max =} \DecValTok{10}\NormalTok{, }\AttributeTok{col\_names =} \ConstantTok{TRUE}\NormalTok{)}
\end{Highlighting}
\end{Shaded}

\begin{verbatim}
## Rows: 10 Columns: 13
## -- Column specification --------------------------------------------------------
## Delimiter: "\t"
## chr  (3): Группа, Пол, Группа крови
## dbl (10): Возраст, Рост, Базофилы_E1, Эозинофилы_E1, Гемоглобин_E1, Эритроци...
## 
## i Use `spec()` to retrieve the full column specification for this data.
## i Specify the column types or set `show_col_types = FALSE` to quiet this message.
\end{verbatim}

\begin{verbatim}
## # A tibble: 10 x 13
##    Группа   Возраст Пол      Рост `Группа крови` Базофилы_E1 Эозинофилы_E1
##    <chr>      <dbl> <chr>   <dbl> <chr>                <dbl>         <dbl>
##  1 Группа 1      31 Женский   174 A (II)              0.422          0.646
##  2 Группа 1      28 Женский   157 A (II)              0.327          4.97 
##  3 Группа 1      33 Женский   166 <NA>                0.799          3.39 
##  4 Группа 1      26 Женский   168 O (I)               0.0237         4.54 
##  5 Группа 1      33 Женский   170 A (II)              0.664          3.32 
##  6 Группа 1      28 Мужской   172 B (III)             0.481          2.79 
##  7 Группа 1      27 Мужской   157 A (II)              0.890          2.34 
##  8 Группа 1      31 Мужской   174 <NA>                0.858          3.98 
##  9 Группа 1      23 Женский   175 A (II)              0.383          3.39 
## 10 Группа 1      29 Женский   172 A (II)              0.281          7.94 
## # i 6 more variables: Гемоглобин_E1 <dbl>, Эритроциты_E1 <dbl>,
## #   Базофилы_E2 <dbl>, Эозинофилы_E2 <dbl>, Гемоглобин_E2 <dbl>,
## #   Эритроциты_E2 <dbl>
\end{verbatim}

\#\#\#iris

\begin{Shaded}
\begin{Highlighting}[]
\FunctionTok{library}\NormalTok{(datasets)}
\FunctionTok{data}\NormalTok{(iris)}
\FunctionTok{summary}\NormalTok{(iris)}
\end{Highlighting}
\end{Shaded}

\begin{verbatim}
##   Sepal.Length    Sepal.Width     Petal.Length    Petal.Width   
##  Min.   :4.300   Min.   :2.000   Min.   :1.000   Min.   :0.100  
##  1st Qu.:5.100   1st Qu.:2.800   1st Qu.:1.600   1st Qu.:0.300  
##  Median :5.800   Median :3.000   Median :4.350   Median :1.300  
##  Mean   :5.843   Mean   :3.057   Mean   :3.758   Mean   :1.199  
##  3rd Qu.:6.400   3rd Qu.:3.300   3rd Qu.:5.100   3rd Qu.:1.800  
##  Max.   :7.900   Max.   :4.400   Max.   :6.900   Max.   :2.500  
##        Species  
##  setosa    :50  
##  versicolor:50  
##  virginica :50  
##                 
##                 
## 
\end{verbatim}

\begin{Shaded}
\begin{Highlighting}[]
\CommentTok{\#write\_csv(data, "data/raw/data\_csv.csv")}

\CommentTok{\#write\_excel\_csv(data, "data/raw/data\_csv.csv")}

\CommentTok{\#write\_csv2(data, "data/raw/data\_csv2.csv")}

\CommentTok{\#write\_excel\_csv2(data, "data/raw/data\_csv2.csv")}

\CommentTok{\#Ничего не работает...}
\end{Highlighting}
\end{Shaded}

\#\#\#read\_excel

\begin{Shaded}
\begin{Highlighting}[]
\FunctionTok{read\_excel}\NormalTok{(}\StringTok{"data/raw/data\_excel.xlsx"}\NormalTok{, }\AttributeTok{sheet =} \StringTok{"data\_csv2"}\NormalTok{)}
\end{Highlighting}
\end{Shaded}

\begin{verbatim}
## # A tibble: 100 x 13
##    Группа   Возраст Пол      Рост `Группа крови` Базофилы_E1 Эозинофилы_E1
##    <chr>      <dbl> <chr>   <dbl> <chr>          <chr>       <chr>        
##  1 Группа 1      31 Женский   174 A (II)         0,4222      0,6465       
##  2 Группа 1      28 Женский   157 A (II)         0,3270      4,9742       
##  3 Группа 1      33 Женский   166 NA             0,7994      3,3875       
##  4 Группа 1      26 Женский   168 O (I)          0,0237      4,5403       
##  5 Группа 1      33 Женский   170 A (II)         0,6636      3,3159       
##  6 Группа 1      28 Мужской   172 B (III)        0,4810      2,7863       
##  7 Группа 1      27 Мужской   157 A (II)         0,8899      2,3432       
##  8 Группа 1      31 Мужской   174 NA             0,8576      3,9788       
##  9 Группа 1      23 Женский   175 A (II)         0,3832      3,3896       
## 10 Группа 1      29 Женский   172 A (II)         0,2812      7,9352       
## # i 90 more rows
## # i 6 more variables: Гемоглобин_E1 <chr>, Эритроциты_E1 <chr>,
## #   Базофилы_E2 <chr>, Эозинофилы_E2 <chr>, Гемоглобин_E2 <chr>,
## #   Эритроциты_E2 <chr>
\end{verbatim}

\#\#\#xlsx::write.xlsx()

\begin{Shaded}
\begin{Highlighting}[]
\CommentTok{\#Существует несколько пакетов, позволяющих записывать данные в книги Excel: xlsx, openxlsx, writexl. Однако, здесь мы будем использовать openxlsx, поскольку он не требует установки Java и наиболее просто устанавливается почти на всех компьютерах.}

\CommentTok{\#write.xlsx(data, "data\_excel.xlsx", colNames = TRUE)}

\CommentTok{\#Вот и всё. У нас появляется файл Excel с записанными данными. Перейдём к бонусу для тех, у кого установлен Java.}

\CommentTok{\#openxlsx::write.xlsx(), openxlsx::write.xlsx2()}

\CommentTok{\#Прежде всего отметим, что дублирующая функция с цифрой 2 используется преимущественно для того, чтобы быстро записывать крайне большие датафреймы (более, чем 100 тысяч ячеек).}

\CommentTok{\#write.xlsx(data, "data\_excel.xlsx", sheetName = "data", col.names = TRUE, row.names = TRUE, append = FALSE)}

\CommentTok{\#Функция создаёт книгу Excel, а в ней лист с соответствующим именем. Аргументы col.names и row.names говорят, нужно ли записывать в файл имена столбцов и имена строк соответственно. Аргумент append нужен для того, чтобы...}

\CommentTok{\#write.xlsx(data, "data\_excel.xlsx", sheetName = "data\_2", col.names = TRUE, row.names = TRUE, append = TRUE)}

\CommentTok{\#...добавлять новые листы к уже существующей книге. }

\CommentTok{\#haven::read\_spss(), haven::read\_sas()}

\CommentTok{\#Очень редко, но приходится читать данные, которые выгружены напрямую из SPSS или SAS.}

\CommentTok{\#haven::write\_sav() }

\CommentTok{\#Точно так же можно записать данные, чтобы наш предполагаемый коллега мог загрузить их в SPSS.}
\end{Highlighting}
\end{Shaded}

\#\#\#mean()

\begin{Shaded}
\begin{Highlighting}[]
\FunctionTok{mean}\NormalTok{(}\FunctionTok{c}\NormalTok{(}\DecValTok{20}\NormalTok{, }\DecValTok{68}\NormalTok{, }\DecValTok{45}\NormalTok{, }\DecValTok{76}\NormalTok{, }\DecValTok{41}\NormalTok{, }\DecValTok{36}\NormalTok{, }\DecValTok{13}\NormalTok{, }\DecValTok{52}\NormalTok{, }\DecValTok{77}\NormalTok{, }\DecValTok{53}\NormalTok{, }\DecValTok{70}\NormalTok{, }\DecValTok{73}\NormalTok{))}
\end{Highlighting}
\end{Shaded}

\begin{verbatim}
## [1] 52
\end{verbatim}

\begin{Shaded}
\begin{Highlighting}[]
\NormalTok{a1 }\OtherTok{\textless{}{-}} \FunctionTok{c}\NormalTok{(}\DecValTok{1}\NormalTok{, }\SpecialCharTok{{-}}\DecValTok{1}\NormalTok{, }\DecValTok{5}\NormalTok{, }\SpecialCharTok{{-}}\DecValTok{12}\NormalTok{, }\SpecialCharTok{{-}}\DecValTok{12}\NormalTok{, }\DecValTok{3}\NormalTok{, }\DecValTok{8}\NormalTok{, }\SpecialCharTok{{-}}\DecValTok{10}\NormalTok{, }\DecValTok{0}\NormalTok{)}
\NormalTok{a2 }\OtherTok{\textless{}{-}} \FunctionTok{c}\NormalTok{(}\DecValTok{76}\NormalTok{, }\DecValTok{65}\NormalTok{, }\DecValTok{71}\NormalTok{, }\DecValTok{16}\NormalTok{, }\DecValTok{60}\NormalTok{, }\DecValTok{29}\NormalTok{, }\DecValTok{71}\NormalTok{, }\DecValTok{46}\NormalTok{, }\DecValTok{45}\NormalTok{, }\DecValTok{41}\NormalTok{)}
\NormalTok{a3 }\OtherTok{\textless{}{-}} \FunctionTok{c}\NormalTok{(}\SpecialCharTok{{-}}\DecValTok{2}\NormalTok{, }\DecValTok{16}\NormalTok{, }\SpecialCharTok{{-}}\DecValTok{3}\NormalTok{, }\DecValTok{16}\NormalTok{, }\SpecialCharTok{{-}}\DecValTok{9}\NormalTok{, }\DecValTok{7}\NormalTok{, }\DecValTok{31}\NormalTok{)}
\NormalTok{a4 }\OtherTok{\textless{}{-}} \FunctionTok{c}\NormalTok{(}\SpecialCharTok{{-}}\DecValTok{19}\NormalTok{, }\SpecialCharTok{{-}}\DecValTok{9}\NormalTok{, }\DecValTok{19}\NormalTok{, }\DecValTok{5}\NormalTok{, }\SpecialCharTok{{-}}\DecValTok{14}\NormalTok{, }\DecValTok{0}\NormalTok{, }\DecValTok{34}\NormalTok{, }\SpecialCharTok{{-}}\DecValTok{8}\NormalTok{, }\DecValTok{34}\NormalTok{, }\DecValTok{24}\NormalTok{, }\SpecialCharTok{{-}}\DecValTok{11}\NormalTok{, }\DecValTok{8}\NormalTok{, }\DecValTok{33}\NormalTok{, }\DecValTok{12}\NormalTok{, }\SpecialCharTok{{-}}\DecValTok{6}\NormalTok{)}
\NormalTok{a5 }\OtherTok{\textless{}{-}} \FunctionTok{c}\NormalTok{(}\ConstantTok{NA}\NormalTok{, }\ConstantTok{NA}\NormalTok{, }\ConstantTok{NA}\NormalTok{, }\ConstantTok{NA}\NormalTok{, }\ConstantTok{NA}\NormalTok{, }\ConstantTok{NA}\NormalTok{, }\DecValTok{3}\NormalTok{, }\ConstantTok{NA}\NormalTok{, }\ConstantTok{NA}\NormalTok{)}
\NormalTok{a6 }\OtherTok{\textless{}{-}} \FunctionTok{c}\NormalTok{(}\SpecialCharTok{{-}}\DecValTok{13}\NormalTok{, }\DecValTok{19}\NormalTok{, }\SpecialCharTok{{-}}\DecValTok{24}\NormalTok{, }\ConstantTok{NA}\NormalTok{, }\DecValTok{30}\NormalTok{, }\DecValTok{64}\NormalTok{, }\SpecialCharTok{{-}}\DecValTok{53}\NormalTok{, }\ConstantTok{NA}\NormalTok{, }\DecValTok{50}\NormalTok{, }\DecValTok{31}\NormalTok{, }\SpecialCharTok{{-}}\DecValTok{58}\NormalTok{, }\SpecialCharTok{{-}}\DecValTok{34}\NormalTok{, }\SpecialCharTok{{-}}\DecValTok{3}\NormalTok{, }\SpecialCharTok{{-}}\DecValTok{34}\NormalTok{, }\DecValTok{77}\NormalTok{)}

\FunctionTok{mean}\NormalTok{(a1)}
\end{Highlighting}
\end{Shaded}

\begin{verbatim}
## [1] -2
\end{verbatim}

\begin{Shaded}
\begin{Highlighting}[]
\FunctionTok{mean}\NormalTok{(a2)}
\end{Highlighting}
\end{Shaded}

\begin{verbatim}
## [1] 52
\end{verbatim}

\begin{Shaded}
\begin{Highlighting}[]
\FunctionTok{mean}\NormalTok{(a3)}
\end{Highlighting}
\end{Shaded}

\begin{verbatim}
## [1] 8
\end{verbatim}

\begin{Shaded}
\begin{Highlighting}[]
\FunctionTok{mean}\NormalTok{(a4)}
\end{Highlighting}
\end{Shaded}

\begin{verbatim}
## [1] 6.8
\end{verbatim}

\begin{Shaded}
\begin{Highlighting}[]
\FunctionTok{mean}\NormalTok{(a5)}
\end{Highlighting}
\end{Shaded}

\begin{verbatim}
## [1] NA
\end{verbatim}

\#\#\#median()

\begin{Shaded}
\begin{Highlighting}[]
\NormalTok{b1 }\OtherTok{\textless{}{-}} \FunctionTok{c}\NormalTok{(}\SpecialCharTok{{-}}\DecValTok{92}\NormalTok{, }\SpecialCharTok{{-}}\DecValTok{50}\NormalTok{, }\DecValTok{54}\NormalTok{, }\DecValTok{55}\NormalTok{, }\DecValTok{84}\NormalTok{, }\DecValTok{52}\NormalTok{, }\SpecialCharTok{{-}}\DecValTok{55}\NormalTok{, }\SpecialCharTok{{-}}\DecValTok{23}\NormalTok{, }\DecValTok{36}\NormalTok{, }\SpecialCharTok{{-}}\DecValTok{11}\NormalTok{, }\DecValTok{22}\NormalTok{, }\DecValTok{11}\NormalTok{, }\SpecialCharTok{{-}}\DecValTok{7}\NormalTok{)}
\NormalTok{b2 }\OtherTok{\textless{}{-}} \FunctionTok{c}\NormalTok{(}\DecValTok{1}\NormalTok{, }\DecValTok{9}\NormalTok{, }\ConstantTok{NA}\NormalTok{, }\DecValTok{88}\NormalTok{, }\DecValTok{2}\NormalTok{, }\ConstantTok{NA}\NormalTok{, }\DecValTok{42}\NormalTok{, }\ConstantTok{NA}\NormalTok{, }\DecValTok{4}\NormalTok{, }\DecValTok{68}\NormalTok{, }\ConstantTok{NA}\NormalTok{)}
\NormalTok{b3 }\OtherTok{\textless{}{-}} \FunctionTok{c}\NormalTok{(}\SpecialCharTok{{-}}\DecValTok{15}\NormalTok{, }\DecValTok{71}\NormalTok{, }\DecValTok{77}\NormalTok{, }\DecValTok{36}\NormalTok{, }\DecValTok{66}\NormalTok{, }\SpecialCharTok{{-}}\DecValTok{21}\NormalTok{, }\SpecialCharTok{{-}}\DecValTok{48}\NormalTok{, }\SpecialCharTok{{-}}\DecValTok{8}\NormalTok{)}
\NormalTok{b4 }\OtherTok{\textless{}{-}} \FunctionTok{c}\NormalTok{(}\DecValTok{19}\NormalTok{, }\DecValTok{89}\NormalTok{, }\DecValTok{78}\NormalTok{, }\DecValTok{38}\NormalTok{, }\DecValTok{8}\NormalTok{, }\DecValTok{17}\NormalTok{, }\DecValTok{25}\NormalTok{, }\DecValTok{60}\NormalTok{, }\DecValTok{8}\NormalTok{, }\DecValTok{43}\NormalTok{, }\DecValTok{29}\NormalTok{, }\DecValTok{6}\NormalTok{, }\DecValTok{62}\NormalTok{, }\DecValTok{41}\NormalTok{, }\DecValTok{69}\NormalTok{, }\DecValTok{97}\NormalTok{, }\DecValTok{61}\NormalTok{, }\DecValTok{83}\NormalTok{, }\DecValTok{25}\NormalTok{, }\DecValTok{24}\NormalTok{)}
\NormalTok{b5 }\OtherTok{\textless{}{-}} \FunctionTok{c}\NormalTok{(}\SpecialCharTok{{-}}\DecValTok{91}\NormalTok{, }\SpecialCharTok{{-}}\DecValTok{33}\NormalTok{, }\DecValTok{13}\NormalTok{, }\DecValTok{34}\NormalTok{, }\DecValTok{34}\NormalTok{, }\DecValTok{75}\NormalTok{, }\SpecialCharTok{{-}}\DecValTok{80}\NormalTok{, }\SpecialCharTok{{-}}\DecValTok{35}\NormalTok{, }\SpecialCharTok{{-}}\DecValTok{90}\NormalTok{, }\SpecialCharTok{{-}}\DecValTok{72}\NormalTok{, }\DecValTok{70}\NormalTok{, }\DecValTok{67}\NormalTok{, }\SpecialCharTok{{-}}\DecValTok{100}\NormalTok{, }\SpecialCharTok{{-}}\DecValTok{94}\NormalTok{, }\SpecialCharTok{{-}}\DecValTok{18}\NormalTok{)}

\FunctionTok{median}\NormalTok{(b1)}
\end{Highlighting}
\end{Shaded}

\begin{verbatim}
## [1] 11
\end{verbatim}

\begin{Shaded}
\begin{Highlighting}[]
\FunctionTok{median}\NormalTok{(b2, }\AttributeTok{na.rm =} \ConstantTok{TRUE}\NormalTok{)}
\end{Highlighting}
\end{Shaded}

\begin{verbatim}
## [1] 9
\end{verbatim}

\begin{Shaded}
\begin{Highlighting}[]
\FunctionTok{median}\NormalTok{(b3)}
\end{Highlighting}
\end{Shaded}

\begin{verbatim}
## [1] 14
\end{verbatim}

\begin{Shaded}
\begin{Highlighting}[]
\FunctionTok{median}\NormalTok{(b4)}
\end{Highlighting}
\end{Shaded}

\begin{verbatim}
## [1] 39.5
\end{verbatim}

\begin{Shaded}
\begin{Highlighting}[]
\FunctionTok{median}\NormalTok{(b5)}
\end{Highlighting}
\end{Shaded}

\begin{verbatim}
## [1] -33
\end{verbatim}

\#\#\#min(), max()

\begin{Shaded}
\begin{Highlighting}[]
\NormalTok{c1 }\OtherTok{\textless{}{-}} \FunctionTok{c}\NormalTok{(}\FloatTok{68.92}\NormalTok{, }\FloatTok{44.15}\NormalTok{, }\FloatTok{34.2}\NormalTok{, }\FloatTok{34.12}\NormalTok{, }\FloatTok{37.7}\NormalTok{, }\FloatTok{73.95}\NormalTok{, }\FloatTok{36.9}\NormalTok{, }\FloatTok{59.26}\NormalTok{, }\FloatTok{31.06}\NormalTok{, }\FloatTok{55.79}\NormalTok{, }\FloatTok{73.92}\NormalTok{, }\FloatTok{68.04}\NormalTok{, }\FloatTok{53.73}\NormalTok{, }\FloatTok{90.7}\NormalTok{, }\FloatTok{39.66}\NormalTok{)}
\NormalTok{c2 }\OtherTok{\textless{}{-}} \FunctionTok{c}\NormalTok{(}\FloatTok{90.48}\NormalTok{, }\FloatTok{31.16}\NormalTok{, }\FloatTok{44.4}\NormalTok{, }\FloatTok{21.94}\NormalTok{, }\FloatTok{84.37}\NormalTok{, }\FloatTok{53.15}\NormalTok{, }\FloatTok{81.15}\NormalTok{, }\FloatTok{47.86}\NormalTok{, }\FloatTok{63.23}\NormalTok{, }\FloatTok{46.75}\NormalTok{, }\FloatTok{102.73}\NormalTok{)}
\NormalTok{c3 }\OtherTok{\textless{}{-}} \FunctionTok{c}\NormalTok{(}\FloatTok{48.11}\NormalTok{, }\FloatTok{45.3}\NormalTok{, }\FloatTok{58.42}\NormalTok{, }\FloatTok{51.64}\NormalTok{, }\FloatTok{62.07}\NormalTok{, }\FloatTok{57.26}\NormalTok{, }\FloatTok{49.69}\NormalTok{, }\FloatTok{93.29}\NormalTok{, }\FloatTok{81.18}\NormalTok{, }\FloatTok{44.78}\NormalTok{, }\FloatTok{55.1}\NormalTok{, }\FloatTok{76.74}\NormalTok{, }\FloatTok{58.08}\NormalTok{)}
\NormalTok{c4 }\OtherTok{\textless{}{-}} \FunctionTok{c}\NormalTok{(}\FloatTok{17.24}\NormalTok{, }\FloatTok{35.77}\NormalTok{, }\FloatTok{57.57}\NormalTok{, }\FloatTok{30.15}\NormalTok{, }\FloatTok{43.27}\NormalTok{, }\FloatTok{77.56}\NormalTok{, }\FloatTok{72.19}\NormalTok{, }\FloatTok{40.45}\NormalTok{, }\FloatTok{46.2}\NormalTok{, }\FloatTok{39.92}\NormalTok{)}
\NormalTok{c5 }\OtherTok{\textless{}{-}} \FunctionTok{c}\NormalTok{(}\FloatTok{60.22}\NormalTok{, }\FloatTok{31.91}\NormalTok{, }\FloatTok{72.71}\NormalTok{, }\FloatTok{52.49}\NormalTok{, }\FloatTok{46.21}\NormalTok{, }\FloatTok{60.39}\NormalTok{, }\FloatTok{60.09}\NormalTok{)}

\FunctionTok{min}\NormalTok{(c1, }\AttributeTok{na.rm =} \ConstantTok{FALSE}\NormalTok{)}
\end{Highlighting}
\end{Shaded}

\begin{verbatim}
## [1] 31.06
\end{verbatim}

\begin{Shaded}
\begin{Highlighting}[]
\FunctionTok{max}\NormalTok{(c1, }\AttributeTok{na.rm =} \ConstantTok{FALSE}\NormalTok{)}
\end{Highlighting}
\end{Shaded}

\begin{verbatim}
## [1] 90.7
\end{verbatim}

\begin{Shaded}
\begin{Highlighting}[]
\FunctionTok{min}\NormalTok{(c2, }\AttributeTok{na.rm =} \ConstantTok{FALSE}\NormalTok{)}
\end{Highlighting}
\end{Shaded}

\begin{verbatim}
## [1] 21.94
\end{verbatim}

\begin{Shaded}
\begin{Highlighting}[]
\FunctionTok{max}\NormalTok{(c2, }\AttributeTok{na.rm =} \ConstantTok{FALSE}\NormalTok{)}
\end{Highlighting}
\end{Shaded}

\begin{verbatim}
## [1] 102.73
\end{verbatim}

\begin{Shaded}
\begin{Highlighting}[]
\FunctionTok{min}\NormalTok{(c3, }\AttributeTok{na.rm =} \ConstantTok{FALSE}\NormalTok{)}
\end{Highlighting}
\end{Shaded}

\begin{verbatim}
## [1] 44.78
\end{verbatim}

\begin{Shaded}
\begin{Highlighting}[]
\FunctionTok{max}\NormalTok{(c3, }\AttributeTok{na.rm =} \ConstantTok{FALSE}\NormalTok{)}
\end{Highlighting}
\end{Shaded}

\begin{verbatim}
## [1] 93.29
\end{verbatim}

\begin{Shaded}
\begin{Highlighting}[]
\FunctionTok{min}\NormalTok{(c4, }\AttributeTok{na.rm =} \ConstantTok{FALSE}\NormalTok{)}
\end{Highlighting}
\end{Shaded}

\begin{verbatim}
## [1] 17.24
\end{verbatim}

\begin{Shaded}
\begin{Highlighting}[]
\FunctionTok{max}\NormalTok{(c4, }\AttributeTok{na.rm =} \ConstantTok{FALSE}\NormalTok{)}
\end{Highlighting}
\end{Shaded}

\begin{verbatim}
## [1] 77.56
\end{verbatim}

\begin{Shaded}
\begin{Highlighting}[]
\FunctionTok{min}\NormalTok{(c5, }\AttributeTok{na.rm =} \ConstantTok{FALSE}\NormalTok{)}
\end{Highlighting}
\end{Shaded}

\begin{verbatim}
## [1] 31.91
\end{verbatim}

\begin{Shaded}
\begin{Highlighting}[]
\FunctionTok{max}\NormalTok{(c5, }\AttributeTok{na.rm =} \ConstantTok{FALSE}\NormalTok{)}
\end{Highlighting}
\end{Shaded}

\begin{verbatim}
## [1] 72.71
\end{verbatim}

\#\#\#quantile()

\begin{Shaded}
\begin{Highlighting}[]
\NormalTok{d1 }\OtherTok{\textless{}{-}} \FunctionTok{c}\NormalTok{(}\FloatTok{80.94}\NormalTok{, }\FloatTok{44.46}\NormalTok{, }\FloatTok{46.33}\NormalTok{, }\FloatTok{65.1}\NormalTok{, }\FloatTok{66.42}\NormalTok{, }\FloatTok{104.43}\NormalTok{, }\FloatTok{53.15}\NormalTok{, }\FloatTok{48.41}\NormalTok{, }\FloatTok{12.88}\NormalTok{, }\FloatTok{51.1}\NormalTok{, }\FloatTok{43.03}\NormalTok{, }\FloatTok{40.3}\NormalTok{, }\FloatTok{33.71}\NormalTok{, }\FloatTok{55.1}\NormalTok{, }\FloatTok{22.17}\NormalTok{)}
\NormalTok{d2 }\OtherTok{\textless{}{-}} \FunctionTok{c}\NormalTok{(}\FloatTok{26.17}\NormalTok{, }\FloatTok{97.73}\NormalTok{, }\FloatTok{24.81}\NormalTok{, }\FloatTok{53.62}\NormalTok{, }\FloatTok{87.72}\NormalTok{, }\FloatTok{45.19}\NormalTok{, }\FloatTok{45.7}\NormalTok{, }\FloatTok{69.63}\NormalTok{, }\FloatTok{36.76}\NormalTok{, }\FloatTok{7.17}\NormalTok{)}
\NormalTok{d3 }\OtherTok{\textless{}{-}} \FunctionTok{c}\NormalTok{(}\FloatTok{63.92}\NormalTok{, }\FloatTok{35.85}\NormalTok{, }\FloatTok{26.9}\NormalTok{, }\FloatTok{48.92}\NormalTok{, }\FloatTok{43.1}\NormalTok{, }\FloatTok{66.94}\NormalTok{, }\FloatTok{47.06}\NormalTok{, }\FloatTok{56.54}\NormalTok{, }\FloatTok{29.1}\NormalTok{, }\FloatTok{58.88}\NormalTok{)}
\NormalTok{d4 }\OtherTok{\textless{}{-}} \FunctionTok{c}\NormalTok{(}\FloatTok{32.05}\NormalTok{, }\FloatTok{93.85}\NormalTok{, }\FloatTok{85.52}\NormalTok{, }\FloatTok{56.69}\NormalTok{, }\FloatTok{23.69}\NormalTok{, }\FloatTok{11.29}\NormalTok{, }\FloatTok{51.44}\NormalTok{, }\FloatTok{63.09}\NormalTok{, }\FloatTok{65.65}\NormalTok{, }\FloatTok{35.73}\NormalTok{, }\FloatTok{60.15}\NormalTok{, }\FloatTok{30.93}\NormalTok{, }\SpecialCharTok{{-}}\FloatTok{4.2}\NormalTok{)}

\FunctionTok{quantile}\NormalTok{(d1, }\AttributeTok{probs =} \FunctionTok{seq}\NormalTok{(}\DecValTok{0}\NormalTok{, }\DecValTok{1}\NormalTok{, }\FloatTok{0.25}\NormalTok{), }\AttributeTok{na.rm =} \ConstantTok{FALSE}\NormalTok{, }\AttributeTok{names =} \ConstantTok{TRUE}\NormalTok{, }\AttributeTok{type =} \DecValTok{7}\NormalTok{)}
\end{Highlighting}
\end{Shaded}

\begin{verbatim}
##      0%     25%     50%     75%    100% 
##  12.880  41.665  48.410  60.100 104.430
\end{verbatim}

\begin{Shaded}
\begin{Highlighting}[]
\FunctionTok{quantile}\NormalTok{(d2, }\AttributeTok{probs =} \FunctionTok{seq}\NormalTok{(}\DecValTok{0}\NormalTok{, }\DecValTok{1}\NormalTok{, }\FloatTok{0.25}\NormalTok{), }\AttributeTok{na.rm =} \ConstantTok{FALSE}\NormalTok{, }\AttributeTok{names =} \ConstantTok{TRUE}\NormalTok{, }\AttributeTok{type =} \DecValTok{7}\NormalTok{)}
\end{Highlighting}
\end{Shaded}

\begin{verbatim}
##      0%     25%     50%     75%    100% 
##  7.1700 28.8175 45.4450 65.6275 97.7300
\end{verbatim}

\begin{Shaded}
\begin{Highlighting}[]
\FunctionTok{quantile}\NormalTok{(d3, }\AttributeTok{probs =} \FunctionTok{seq}\NormalTok{(}\DecValTok{0}\NormalTok{, }\DecValTok{1}\NormalTok{, }\FloatTok{0.025}\NormalTok{), }\AttributeTok{na.rm =} \ConstantTok{FALSE}\NormalTok{, }\AttributeTok{names =} \ConstantTok{TRUE}\NormalTok{, }\AttributeTok{type =} \DecValTok{7}\NormalTok{)}
\end{Highlighting}
\end{Shaded}

\begin{verbatim}
##       0%     2.5%       5%     7.5%      10%    12.5%      15%    17.5% 
## 26.90000 27.39500 27.89000 28.38500 28.88000 29.94375 31.46250 32.98125 
##      20%    22.5%      25%    27.5%      30%    32.5%      35%    37.5% 
## 34.50000 36.03125 37.66250 39.29375 40.92500 42.55625 43.69400 44.58500 
##      40%    42.5%      45%    47.5%      50%    52.5%      55%    57.5% 
## 45.47600 46.36700 47.15300 47.57150 47.99000 48.40850 48.82700 50.25350 
##      60%    62.5%      65%    67.5%      70%    72.5%      75%    77.5% 
## 51.96800 53.68250 55.39700 56.71550 57.24200 57.76850 58.29500 58.82150 
##      80%    82.5%      85%    87.5%      90%    92.5%      95%    97.5% 
## 59.88800 61.02200 62.15600 63.29000 64.22200 64.90150 65.58100 66.26050 
##     100% 
## 66.94000
\end{verbatim}

\begin{Shaded}
\begin{Highlighting}[]
\FunctionTok{quantile}\NormalTok{(d4, }\AttributeTok{probs =} \FunctionTok{seq}\NormalTok{(}\DecValTok{0}\NormalTok{, }\DecValTok{1}\NormalTok{, }\FloatTok{0.25}\NormalTok{), }\AttributeTok{na.rm =} \ConstantTok{FALSE}\NormalTok{, }\AttributeTok{names =} \ConstantTok{TRUE}\NormalTok{, }\AttributeTok{type =} \DecValTok{7}\NormalTok{)}
\end{Highlighting}
\end{Shaded}

\begin{verbatim}
##    0%   25%   50%   75%  100% 
## -4.20 30.93 51.44 63.09 93.85
\end{verbatim}

\#\#\#var(), sd()

\begin{Shaded}
\begin{Highlighting}[]
\NormalTok{e1 }\OtherTok{\textless{}{-}} \FunctionTok{c}\NormalTok{(}\FloatTok{47.44}\NormalTok{, }\FloatTok{62.44}\NormalTok{, }\FloatTok{20.44}\NormalTok{, }\FloatTok{72.75}\NormalTok{, }\FloatTok{77.86}\NormalTok{, }\FloatTok{13.74}\NormalTok{, }\FloatTok{28.2}\NormalTok{, }\FloatTok{50.47}\NormalTok{, }\FloatTok{59.19}\NormalTok{, }\FloatTok{69.04}\NormalTok{)}
\NormalTok{e2 }\OtherTok{\textless{}{-}} \FunctionTok{c}\NormalTok{(}\FloatTok{49.31}\NormalTok{, }\FloatTok{44.47}\NormalTok{, }\FloatTok{14.04}\NormalTok{, }\FloatTok{44.43}\NormalTok{, }\FloatTok{49.18}\NormalTok{, }\FloatTok{40.73}\NormalTok{, }\FloatTok{44.65}\NormalTok{, }\FloatTok{41.91}\NormalTok{, }\FloatTok{80.38}\NormalTok{, }\FloatTok{80.09}\NormalTok{)}
\NormalTok{e3 }\OtherTok{\textless{}{-}} \FunctionTok{c}\NormalTok{(}\FloatTok{57.96}\NormalTok{, }\FloatTok{20.81}\NormalTok{, }\FloatTok{8.92}\NormalTok{, }\FloatTok{14.03}\NormalTok{, }\FloatTok{61.02}\NormalTok{, }\FloatTok{25.69}\NormalTok{, }\FloatTok{21.22}\NormalTok{, }\FloatTok{49.56}\NormalTok{, }\FloatTok{25.64}\NormalTok{, }\FloatTok{28.31}\NormalTok{)}
\NormalTok{e4 }\OtherTok{\textless{}{-}} \FunctionTok{c}\NormalTok{(}\FloatTok{76.22}\NormalTok{, }\DecValTok{65}\NormalTok{, }\FloatTok{19.69}\NormalTok{, }\FloatTok{29.84}\NormalTok{, }\FloatTok{37.18}\NormalTok{, }\FloatTok{70.93}\NormalTok{, }\FloatTok{64.78}\NormalTok{, }\FloatTok{61.66}\NormalTok{, }\FloatTok{49.03}\NormalTok{, }\FloatTok{51.56}\NormalTok{)}
\NormalTok{e5 }\OtherTok{\textless{}{-}} \FunctionTok{c}\NormalTok{(}\FloatTok{92.11}\NormalTok{, }\DecValTok{56}\NormalTok{, }\FloatTok{47.89}\NormalTok{, }\FloatTok{62.96}\NormalTok{, }\FloatTok{47.41}\NormalTok{, }\FloatTok{37.05}\NormalTok{, }\FloatTok{73.96}\NormalTok{, }\DecValTok{53}\NormalTok{, }\FloatTok{52.37}\NormalTok{, }\FloatTok{85.23}\NormalTok{)}

\FunctionTok{var}\NormalTok{(e1, }\AttributeTok{na.rm =} \ConstantTok{TRUE}\NormalTok{)}
\end{Highlighting}
\end{Shaded}

\begin{verbatim}
## [1] 507.3136
\end{verbatim}

\begin{Shaded}
\begin{Highlighting}[]
\FunctionTok{sd}\NormalTok{(e1, }\AttributeTok{na.rm =} \ConstantTok{TRUE}\NormalTok{)}
\end{Highlighting}
\end{Shaded}

\begin{verbatim}
## [1] 22.52362
\end{verbatim}

\begin{Shaded}
\begin{Highlighting}[]
\FunctionTok{var}\NormalTok{(e2, }\AttributeTok{na.rm =} \ConstantTok{TRUE}\NormalTok{)}
\end{Highlighting}
\end{Shaded}

\begin{verbatim}
## [1] 372.5051
\end{verbatim}

\begin{Shaded}
\begin{Highlighting}[]
\FunctionTok{sd}\NormalTok{(e2, }\AttributeTok{na.rm =} \ConstantTok{TRUE}\NormalTok{)}
\end{Highlighting}
\end{Shaded}

\begin{verbatim}
## [1] 19.30039
\end{verbatim}

\begin{Shaded}
\begin{Highlighting}[]
\FunctionTok{var}\NormalTok{(e3, }\AttributeTok{na.rm =} \ConstantTok{TRUE}\NormalTok{)}
\end{Highlighting}
\end{Shaded}

\begin{verbatim}
## [1] 334.519
\end{verbatim}

\begin{Shaded}
\begin{Highlighting}[]
\FunctionTok{sd}\NormalTok{(e3, }\AttributeTok{na.rm =} \ConstantTok{TRUE}\NormalTok{)}
\end{Highlighting}
\end{Shaded}

\begin{verbatim}
## [1] 18.28986
\end{verbatim}

\begin{Shaded}
\begin{Highlighting}[]
\FunctionTok{var}\NormalTok{(e4, }\AttributeTok{na.rm =} \ConstantTok{TRUE}\NormalTok{)}
\end{Highlighting}
\end{Shaded}

\begin{verbatim}
## [1] 347.8641
\end{verbatim}

\begin{Shaded}
\begin{Highlighting}[]
\FunctionTok{sd}\NormalTok{(e4, }\AttributeTok{na.rm =} \ConstantTok{TRUE}\NormalTok{)}
\end{Highlighting}
\end{Shaded}

\begin{verbatim}
## [1] 18.65111
\end{verbatim}

\begin{Shaded}
\begin{Highlighting}[]
\FunctionTok{var}\NormalTok{(e5, }\AttributeTok{na.rm =} \ConstantTok{TRUE}\NormalTok{)}
\end{Highlighting}
\end{Shaded}

\begin{verbatim}
## [1] 313.3289
\end{verbatim}

\begin{Shaded}
\begin{Highlighting}[]
\FunctionTok{sd}\NormalTok{(e5, }\AttributeTok{na.rm =} \ConstantTok{TRUE}\NormalTok{)}
\end{Highlighting}
\end{Shaded}

\begin{verbatim}
## [1] 17.7011
\end{verbatim}

\#\#\#IQR()

\begin{Shaded}
\begin{Highlighting}[]
\NormalTok{f1 }\OtherTok{\textless{}{-}} \FunctionTok{c}\NormalTok{(}\FloatTok{80.94}\NormalTok{, }\FloatTok{44.46}\NormalTok{, }\FloatTok{46.33}\NormalTok{, }\FloatTok{65.1}\NormalTok{, }\FloatTok{66.42}\NormalTok{, }\FloatTok{104.43}\NormalTok{, }\FloatTok{53.15}\NormalTok{, }\FloatTok{48.41}\NormalTok{, }\FloatTok{12.88}\NormalTok{, }\FloatTok{51.1}\NormalTok{, }\FloatTok{43.03}\NormalTok{, }\FloatTok{40.3}\NormalTok{, }\FloatTok{33.71}\NormalTok{, }\FloatTok{55.1}\NormalTok{, }\FloatTok{22.17}\NormalTok{)}
\NormalTok{f2 }\OtherTok{\textless{}{-}} \FunctionTok{c}\NormalTok{(}\FloatTok{26.17}\NormalTok{, }\FloatTok{97.73}\NormalTok{, }\FloatTok{24.81}\NormalTok{, }\FloatTok{53.62}\NormalTok{, }\FloatTok{87.72}\NormalTok{, }\FloatTok{45.19}\NormalTok{, }\FloatTok{45.7}\NormalTok{, }\FloatTok{69.63}\NormalTok{, }\FloatTok{36.76}\NormalTok{, }\FloatTok{7.17}\NormalTok{)}
\NormalTok{f3 }\OtherTok{\textless{}{-}} \FunctionTok{c}\NormalTok{(}\FloatTok{63.92}\NormalTok{, }\FloatTok{35.85}\NormalTok{, }\FloatTok{26.9}\NormalTok{, }\FloatTok{48.92}\NormalTok{, }\FloatTok{43.1}\NormalTok{, }\FloatTok{66.94}\NormalTok{, }\FloatTok{47.06}\NormalTok{, }\FloatTok{56.54}\NormalTok{, }\FloatTok{29.1}\NormalTok{, }\FloatTok{58.88}\NormalTok{)}
\NormalTok{f4 }\OtherTok{\textless{}{-}} \FunctionTok{c}\NormalTok{(}\FloatTok{32.05}\NormalTok{, }\FloatTok{93.85}\NormalTok{, }\FloatTok{85.52}\NormalTok{, }\FloatTok{56.69}\NormalTok{, }\FloatTok{23.69}\NormalTok{, }\FloatTok{11.29}\NormalTok{, }\FloatTok{51.44}\NormalTok{, }\FloatTok{63.09}\NormalTok{, }\FloatTok{65.65}\NormalTok{, }\FloatTok{35.73}\NormalTok{, }\FloatTok{60.15}\NormalTok{, }\FloatTok{30.93}\NormalTok{, }\SpecialCharTok{{-}}\FloatTok{4.2}\NormalTok{)}

\FunctionTok{IQR}\NormalTok{(f1, }\AttributeTok{na.rm =} \ConstantTok{FALSE}\NormalTok{, }\AttributeTok{type =} \DecValTok{7}\NormalTok{)}
\end{Highlighting}
\end{Shaded}

\begin{verbatim}
## [1] 18.435
\end{verbatim}

\begin{Shaded}
\begin{Highlighting}[]
\FunctionTok{IQR}\NormalTok{(f2, }\AttributeTok{na.rm =} \ConstantTok{FALSE}\NormalTok{, }\AttributeTok{type =} \DecValTok{7}\NormalTok{)}
\end{Highlighting}
\end{Shaded}

\begin{verbatim}
## [1] 36.81
\end{verbatim}

\begin{Shaded}
\begin{Highlighting}[]
\FunctionTok{IQR}\NormalTok{(f3, }\AttributeTok{na.rm =} \ConstantTok{FALSE}\NormalTok{, }\AttributeTok{type =} \DecValTok{7}\NormalTok{)}
\end{Highlighting}
\end{Shaded}

\begin{verbatim}
## [1] 20.6325
\end{verbatim}

\begin{Shaded}
\begin{Highlighting}[]
\FunctionTok{IQR}\NormalTok{(f4, }\AttributeTok{na.rm =} \ConstantTok{FALSE}\NormalTok{, }\AttributeTok{type =} \DecValTok{7}\NormalTok{)}
\end{Highlighting}
\end{Shaded}

\begin{verbatim}
## [1] 32.16
\end{verbatim}

\#\#\#length()

\begin{Shaded}
\begin{Highlighting}[]
\CommentTok{\#Приведём лайфхак, косвенно связанный с количеством значений вектора. }

\CommentTok{\#Зачастую в работе нам нужно включить в таблицы количество значений без пропущенных значений и количество именно пропущенных значений:}

\CommentTok{\#sum(!is.na(vec)): количество значений без учёта пропущенных;}
\CommentTok{\#sum(is.na(vec)): количество пропущенных значений.}
\end{Highlighting}
\end{Shaded}

\#\#\#Стандартная ошибка среднего sd(x)/sqrt(length(x))

\begin{Shaded}
\begin{Highlighting}[]
\NormalTok{d1 }\OtherTok{\textless{}{-}} \FunctionTok{c}\NormalTok{(}\FloatTok{47.44}\NormalTok{, }\FloatTok{62.44}\NormalTok{, }\FloatTok{20.44}\NormalTok{, }\FloatTok{72.75}\NormalTok{, }\FloatTok{77.86}\NormalTok{, }\FloatTok{13.74}\NormalTok{, }\FloatTok{28.2}\NormalTok{, }\FloatTok{50.47}\NormalTok{, }\FloatTok{59.19}\NormalTok{, }\FloatTok{69.04}\NormalTok{)}
\NormalTok{d2 }\OtherTok{\textless{}{-}} \FunctionTok{c}\NormalTok{(}\FloatTok{49.31}\NormalTok{, }\FloatTok{44.47}\NormalTok{, }\FloatTok{14.04}\NormalTok{, }\FloatTok{44.43}\NormalTok{, }\FloatTok{49.18}\NormalTok{, }\FloatTok{40.73}\NormalTok{, }\FloatTok{44.65}\NormalTok{, }\FloatTok{41.91}\NormalTok{, }\FloatTok{80.38}\NormalTok{, }\FloatTok{80.09}\NormalTok{)}
\NormalTok{d3 }\OtherTok{\textless{}{-}} \FunctionTok{c}\NormalTok{(}\FloatTok{57.96}\NormalTok{, }\FloatTok{20.81}\NormalTok{, }\FloatTok{8.92}\NormalTok{, }\FloatTok{14.03}\NormalTok{, }\FloatTok{61.02}\NormalTok{, }\FloatTok{25.69}\NormalTok{, }\FloatTok{21.22}\NormalTok{, }\FloatTok{49.56}\NormalTok{, }\FloatTok{25.64}\NormalTok{, }\FloatTok{28.31}\NormalTok{)}
\NormalTok{d4 }\OtherTok{\textless{}{-}} \FunctionTok{c}\NormalTok{(}\FloatTok{76.22}\NormalTok{, }\DecValTok{65}\NormalTok{, }\FloatTok{19.69}\NormalTok{, }\FloatTok{29.84}\NormalTok{, }\FloatTok{37.18}\NormalTok{, }\FloatTok{70.93}\NormalTok{, }\FloatTok{64.78}\NormalTok{, }\FloatTok{61.66}\NormalTok{, }\FloatTok{49.03}\NormalTok{, }\FloatTok{51.56}\NormalTok{)}
\NormalTok{d5 }\OtherTok{\textless{}{-}} \FunctionTok{c}\NormalTok{(}\FloatTok{92.11}\NormalTok{, }\DecValTok{56}\NormalTok{, }\FloatTok{47.89}\NormalTok{, }\FloatTok{62.96}\NormalTok{, }\FloatTok{47.41}\NormalTok{, }\FloatTok{37.05}\NormalTok{, }\FloatTok{73.96}\NormalTok{, }\DecValTok{53}\NormalTok{, }\FloatTok{52.37}\NormalTok{, }\FloatTok{85.23}\NormalTok{)}

\FunctionTok{sd}\NormalTok{(d1)}\SpecialCharTok{/}\FunctionTok{sqrt}\NormalTok{(}\FunctionTok{length}\NormalTok{(d1))}
\end{Highlighting}
\end{Shaded}

\begin{verbatim}
## [1] 7.122595
\end{verbatim}

\begin{Shaded}
\begin{Highlighting}[]
\FunctionTok{sd}\NormalTok{(d2)}\SpecialCharTok{/}\FunctionTok{sqrt}\NormalTok{(}\FunctionTok{length}\NormalTok{(d2))}
\end{Highlighting}
\end{Shaded}

\begin{verbatim}
## [1] 6.10332
\end{verbatim}

\begin{Shaded}
\begin{Highlighting}[]
\FunctionTok{sd}\NormalTok{(d3)}\SpecialCharTok{/}\FunctionTok{sqrt}\NormalTok{(}\FunctionTok{length}\NormalTok{(d3))}
\end{Highlighting}
\end{Shaded}

\begin{verbatim}
## [1] 5.783761
\end{verbatim}

\begin{Shaded}
\begin{Highlighting}[]
\FunctionTok{sd}\NormalTok{(d4)}\SpecialCharTok{/}\FunctionTok{sqrt}\NormalTok{(}\FunctionTok{length}\NormalTok{(d4))}
\end{Highlighting}
\end{Shaded}

\begin{verbatim}
## [1] 5.898
\end{verbatim}

\begin{Shaded}
\begin{Highlighting}[]
\FunctionTok{sd}\NormalTok{(d5)}\SpecialCharTok{/}\FunctionTok{sqrt}\NormalTok{(}\FunctionTok{length}\NormalTok{(d5))}
\end{Highlighting}
\end{Shaded}

\begin{verbatim}
## [1] 5.597579
\end{verbatim}

\#\#\#summary()

\#\#\#psych::describe()

\begin{Shaded}
\begin{Highlighting}[]
\NormalTok{data }\OtherTok{\textless{}{-}} \FunctionTok{read\_rds}\NormalTok{(}\StringTok{"data/raw/numeric\_data.rds"}\NormalTok{)}
\FunctionTok{describe}\NormalTok{(data, }\AttributeTok{na.rm =} \ConstantTok{TRUE}\NormalTok{, }\AttributeTok{skew =} \ConstantTok{FALSE}\NormalTok{, }\AttributeTok{ranges =} \ConstantTok{TRUE}\NormalTok{)}
\end{Highlighting}
\end{Shaded}

\begin{verbatim}
##               vars   n   mean   sd median    min    max range   se
## Возраст          1 100  30.25 3.98  30.50  21.00  42.00 21.00 0.40
## Рост             2 100 167.70 5.77 168.00 155.00 181.00 26.00 0.58
## Базофилы_E1      3 100   0.65 0.38   0.65  -0.22   1.72  1.94 0.04
## Эозинофилы_E1    4 100   3.71 2.15   3.73  -1.23   8.43  9.66 0.21
## Гемоглобин_E1    5 100  11.86 1.78  11.71   5.35  16.23 10.88 0.18
## Эритроциты_E1    6 100   4.10 0.67   4.08   2.82   5.73  2.91 0.07
## Базофилы_E2      7 100   1.06 0.38   1.06   0.19   2.12  1.94 0.04
## Эозинофилы_E2    8 100   4.72 2.15   4.74  -0.21   9.45  9.66 0.21
## Гемоглобин_E2    9 100  12.58 1.78  12.43   6.07  16.95 10.88 0.18
## Эритроциты_E2   10 100   6.42 0.67   6.40   5.14   8.04  2.91 0.07
\end{verbatim}

\#\#\#table(), prop.table()

\begin{Shaded}
\begin{Highlighting}[]
\NormalTok{data1 }\OtherTok{\textless{}{-}} \FunctionTok{read\_rds}\NormalTok{(}\StringTok{"data/raw/factor\_data.rds"}\NormalTok{)}
\FunctionTok{table}\NormalTok{(data1}\SpecialCharTok{$}\NormalTok{Группа, data1}\SpecialCharTok{$}\StringTok{\textasciigrave{}}\AttributeTok{Группа крови}\StringTok{\textasciigrave{}}\NormalTok{)}
\end{Highlighting}
\end{Shaded}

\begin{verbatim}
##                             
##                              O (I) A (II) B (III) AB (IV) <NA>
##   \xc3\xf0\xf3\xef\xef\xe0 1    10     16      11       4    9
##   \xc3\xf0\xf3\xef\xef\xe0 2    15     18       5       4    8
\end{verbatim}

\#\#\#tibble()

\begin{Shaded}
\begin{Highlighting}[]
\CommentTok{\#Особенности tibble:}

\CommentTok{\#tibble не изменяют тип ввода. Если вы вводили переменные типа character, то такими они и будут. Если numeric, то будут numeric. В том числе это позволяет даже вводить в качестве значений ячеек списки!}
\CommentTok{\#tibble не меняет имена переменных: если вы ввели имя с пробелом, то он не будет заполнен точкой или иным знаком, однако, нужно оборачивать имена в апострофы (чаще всего находится на клавише буквы "ё" в верхнем левом углу клавиатуры). Это следует делать всякий раз, когда имя переменной отличается от простой строки на латинице без иных знаков. Например: columnname не требует апострофов, а \textasciigrave{}column name\textasciigrave{} уже требует;}
\CommentTok{\#Оценивает аргументы лениво и последовательно, что мы разберём в следующем шаге;}
\CommentTok{\#Не использует имена строк;}
\CommentTok{\#При выводе таблицы данных в печать автоматически будут показаны только первые 10 строк и все столбцы, которые поместятся на экран. Благодаря этому даже очень большой датафрейм не заставит наш компьютер зависнуть.}
\CommentTok{\#Заметка: чтобы превратить data.frame в tibble, достаточно просто применить функцию as\_tibble()}
\end{Highlighting}
\end{Shaded}

tibble(var\_first = 1:10, var\_second = ifelse(var\_first \textless{} 5,
var\_first + 100, var\_first)) tibble(var = 1:10, var = var - 10000000)
tibble(\texttt{var\ 1} = 1:10, \texttt{var\ 2} = \texttt{var\ 1} * 100)
tibble(var\_first = 1:10, var\_first = ifelse(var\_first \textless{} 5,
var\_first + 100, var\_first)) tibble(\texttt{var\ 2} = 10:1,
\texttt{var\ 3} = \texttt{var\ 1} - 10) tibble(var\_1 = c(1:10) - 100,
var\_2 = 1:100)

\#\#\#View() \#\#\#tibble::add\_column()

\begin{Shaded}
\begin{Highlighting}[]
\CommentTok{\#data \%\textgreater{}\% add\_column(column\_name = 1:10, .before = NULL, .after = NULL)}

\CommentTok{\#data: просто имя датафрейма, к которому мы хотим добавить столбец;}
\CommentTok{\#column\_name: это имя нового столбца. Оно может быть любым, не только таким, как в примере;}
\CommentTok{\#.before: номер уже существующего столбца, перед которым нужно поставить новый;}
\CommentTok{\#.after: то же, но уже после которого нужно поставить новый. Хитрый приём: если нужно поставить переменную в конец датафрейма, то в значение можно поставить Inf.}
\end{Highlighting}
\end{Shaded}

\#\#\#tibble::add\_row()

\begin{Shaded}
\begin{Highlighting}[]
\CommentTok{\#data \%\textgreater{}\% add\_row(var\_1 = 1, var\_2 = "value", .before = NULL, .after = NULL)}

\CommentTok{\#data: просто имя датафрейма, к которому мы хотим добавить столбец;}
\CommentTok{\#var\_1, var\_2: это имя нового столбца. Оно может быть любым, не только таким, как в примере;}
\CommentTok{\#.before: номер уже существующей строки, перед которым нужно поставить новый;}
\CommentTok{\#.after: то же, но уже после которого нужно поставить новый. Хитрый приём: если нужно поставить переменную в низ датафрейма, то в значение можно поставить Inf}
\end{Highlighting}
\end{Shaded}

\#\#\#dplyr::row\_number()

\#\#\#dplyr::bind\_cols()

\begin{Shaded}
\begin{Highlighting}[]
\CommentTok{\#Мы можем сделать из двух и более таблиц одну, склеив их столбцы.}

\CommentTok{\#data\_1 \%\textgreater{}\% bind\_cols(data\_2) \%\textgreater{}\% bind\_cols(data\_3)}
\end{Highlighting}
\end{Shaded}


\end{document}
